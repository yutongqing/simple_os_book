\subsubsection{操作系统抽象}

接下来读者可站在操作系统实现的角度来看操作系统。操作系统为了能够更好地管理计算机系统并对应用程序提供便捷的服务,在操作系统的发展过程中,计算机科学家提出了如下四个个抽象概念,奠定了操作系统内核设计与实现的基础。操作系统原理中的其他基本概念基本上都基于上述这四个操作系统抽象。

\paragraph{中断(Interrupt)}

简单地说,中断是处理器在执行过程中的突变,用来响应处理器状态中的特殊变化。比如当应用程序正在执行时,产生了时钟外设中断,导致操作系统打断当前应用程序的执行,转而去处理时钟外设中断,处理完毕后,再回到应用程序被打断的地方继续执行。在操作系统中,有三类中断:外设中断(Device Interrupt)、陷阱中断(Trap Interrupt)和故障中断(Fault Interrupt,也称为exception,异常)。外设中断由外部设备引起的外部I/O事件如时钟中断、控制台中断等。外设中断是异步产生的,与处理器的执行无关。故障中断是在处理器执行指令期间检测到不正常的或非法的内部事件(如除零错、地址访问越界)。陷阱中断是在程序中使用请求操作系统服务的系统调用而引发的有意事件。在后面的叙述中,如果没有特别指出,我们将用简称中断、陷阱、故障来区分这三种特殊的中断事件,在不需要区分的地方,统一用中断表示。

\paragraph{进程(Process)}

简单地说,进程是一个正在运行的程序。在计算机系统中,我们可以“同时”运行多个程序,这个“同时”,其实是操作系统给用户造成的一个“幻觉”。大家知道,处理器是计算机系统中的硬件资源。为了提高处理器的利用率,操作系统采用了多道程序技术。如果一个程序因某个事件而不能运行下去时,就把处理器占用权转交给另一个可运行程序。为了刻画多道程序的并发执行的过程,就要引入进程的概念。从操作系统原理上看,一个进程是一个具有一定独立功能的程序在一个数据集合上的一次动态执行过程。操作系统中的进程管理需要协调多道程序之间的关系,解决对处理器分配调度策略、分配实施和回收等问题,从而使得处理器资源得到最充分的利用。

\paragraph{虚存(Virtual Memory)}

简单地说,虚存就是操作系统通过处理器中的MMU硬件的支持而给应用程序和用户提供一个大的(超过计算机中的内存条容量)、一致的(连续的地址空间)、私有的(其他应用程序无法破坏)的存储空间。这需要操作系统将内存和硬盘结合起来管理,为用户提供一个容量比实际内存大得多的虚拟存储器,并且需要操作系统为应用程序分配内存空间,使用户存放在内存中的程序和数据彼此隔离、互不侵扰。操作系统中的虚存管理与处理器的MMU密切相关。

\paragraph{文件(File)}

简单地说,文件就是存放在持久存储介质(比如硬盘、光盘、U盘等)上,方便应用程序和用户读写的数据。当处理器需要访问文件中的数据时,可通过操作系统把它们装入内存。放在硬盘上的程序也是一种文件。文件管理的任务是有效地支持文件的存储、检索和修改等操作。
