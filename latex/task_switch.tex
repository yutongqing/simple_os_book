\section{【背景】80386的任务切换}\label{ux80ccux666f80386ux7684ux4efbux52a1ux5207ux6362}

\lstinline!任务!是80386硬件描述中的一个名词,在这里我们可以简单地把运行在内核态的ucore成为一个任务,把运行在用户态的应用称为另外一任务。任务寄存器(Task
Register,简称TR)
储存了一个16位的选择子(对软件可见),用来索引全局描述符表(GDT)中的一项。TR对应的描述符描述的一个任务状态段(TSS:Task
Status Segment)。

TSS 任务状态段(Task State
Segment,简称TSS)。任务状态段(TSS)是位于GDT中的一个系统段描述符。任务状态段是做什么的呢?任务状态段就是内存中的一个数据结构。这个结构中保存着和任务相关的信息。当发生任务切换的时候会把当前任务用到的寄存器内容(CS/
EIP/
DS/SS/EFLAGS\ldots{})等保存在TSS中以便任务切换回来时候继续使用。ucore根据80386硬件手册建立的TSS数据结构如下所示:

\begin{lstlisting}
struct taskstate {                                                                                 
    uint32_t ts_link;       // 链接字段
    uintptr_t ts_esp0;      // 0级栈指针
    uint16_t ts_ss0;        // 0级栈段寄存器
    uint16_t ts_padding1;
    uintptr_t ts_esp1;
    uint16_t ts_ss1;
    uint16_t ts_padding2;
    uintptr_t ts_esp2;
    uint16_t ts_ss2;
    uint16_t ts_padding3;
    physaddr_t ts_cr3;      // 页目录基址寄存器
    uintptr_t ts_eip;       // 切换的上次EIP
    uint32_t ts_eflags;
    uint32_t ts_eax;        // 保存的通用寄存器eax
    uint32_t ts_ecx;
    uint32_t ts_edx;
    uint32_t ts_ebx;
    uintptr_t ts_esp;
    uintptr_t ts_ebp;
    uint32_t ts_esi;
    uint32_t ts_edi;
    uint16_t ts_es;         // 保存的段寄存器
    uint16_t ts_padding4;
    uint16_t ts_cs;
    uint16_t ts_padding5;
    uint16_t ts_ss;
    uint16_t ts_padding6;
    uint16_t ts_ds;
    uint16_t ts_padding7;
    uint16_t ts_fs;
    uint16_t ts_padding8;
    uint16_t ts_gs;
    uint16_t ts_padding9;
    uint16_t ts_ldt;
    uint16_t ts_padding10;
    uint16_t ts_t;          // 调试陷阱标志(只用位0)
    uint16_t ts_iomb;       // i/o map 基地址
};
\end{lstlisting}

从上图中可以
,TSS的基本格式由104字节组成。这104字节的基本格式是不可改变的,但在此之外系统软件还可定义若干附加信息。基本的104字节可分为链接字段区域、内层栈指针区域、地址映射寄存器区域、寄存器保存区域和其它字段等五个区域。

其中比较重要的是内层栈指针区域,为了有效地实现保护,同一个任务在不同的特权级下使用不同的栈。例如,当从外层特权级3变换到内层特权级0时,任务使用的栈也同时从3级变换到0级栈;当从内层特权级0变换到外层特权级3时,任务使用的栈也同时从0级栈变换到3级栈。所以ucore使用的是0级栈,用户态应用使用的是3级栈。
TSS的内层栈指针区域中有三个栈指针,它们都是48位的全指针(16位的选择子和32位的偏移),分别指向0级、1级和2级栈的栈顶,依次存放在TSS中偏移为4、12及20开始的位置。当发生从3级向0级转移时,把0级栈指针装入0级的SS及ESP寄存器以变换到0级栈。没有指向3级栈的指针,因为3级是最外层,所以任何一个向内层的转移都不可能转移到3级。但是,当特权级由0级向3级变换时,并不把0级栈的指针保存到TSS的栈指针区域。这表明向3级向0级转移时,总是把0级栈认为是一个空栈。

当发生任务切换时,80386中各寄存器的当前值被自动保存到TR所指定的TSS中,然后下一任务的TSS的选择子被装入TR;最后,从TR所指定的TSS中取出各寄存器的值送到处理器的各寄存器中。由此可见,通过在TSS中保存任务现场各寄存器状态的完整映象,实现任务的切换。
