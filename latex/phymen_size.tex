\section{【背景】探测计算机系统中的物理内存分布和大小}\label{ux80ccux666fux63a2ux6d4bux8ba1ux7b97ux673aux7cfbux7edfux4e2dux7684ux7269ux7406ux5185ux5b58ux5206ux5e03ux548cux5927ux5c0f}

在proj5中,操作系统需要知道了解整个计算机系统中的物理内存如何分布的,哪些被可用,哪些不可用。其基本方法是通过BIOS中断调用来帮助完成的。其中BIOS中断调用必须在实模式下进行,所以在bootloader进入保护模式前完成这部分工作相对比较合适。这些部分由boot/bootasm.S中从probe\_memory处到finish\_probe处的代码部分完成完成。通过BIOS中断获取内存可调用参数为e820h的INT
15h BIOS中断。BIOS通过系统内存映射地址描述符(Address Range
Descriptor)格式来表示系统物理内存布局,其具体表示如下:

\begin{lstlisting}
Offset  Size    Description     
00h    8字节   base address               #系统内存块基地址
08h    8字节   length in bytes            #系统内存大小
10h    4字节   type of address range  #内存类型
\end{lstlisting}

看下面的(Values for System Memory Map address type) Values for System
Memory Map address type:

\begin{lstlisting}
01h    memory, available to OS
02h    reserved, not available (e.g. system ROM, memory-mapped device)
03h    ACPI Reclaim Memory (usable by OS after reading ACPI tables)
04h    ACPI NVS Memory (OS is required to save this memory between NVS sessions)
other  not defined yet -- treat as Reserved
\end{lstlisting}

INT15h BIOS中断的详细调用参数:

\begin{lstlisting}
eax:e820h:INT 15的中断调用参数;
edx:534D4150h (即4个ASCII字符“SMAP”) ,这只是一个签名而已;
ebx:如果是第一次调用或内存区域扫描完毕,则为0。 如果不是,则存放上次调用之后的计数值;
ecx:保存地址范围描述符的内存大小,应该大于等于20字节;
es:di:指向保存地址范围描述符结构的缓冲区,BIOS把信息写入这个结构的起始地址。
\end{lstlisting}

此中断的返回值为:

\begin{lstlisting}
cflags的CF位:若INT 15中断执行成功,则不置位,否则置位; 
eax:534D4150h ('SMAP') ;
es:di:指向保存地址范围描述符的缓冲区,此时缓冲区内的数据已由BIOS填写完毕 
ebx:下一个地址范围描述符的计数地址 
ecx :返回BIOS往ES:DI处写的地址范围描述符的字节大小 
ah:失败时保存出错代码
\end{lstlisting}

这样,我们通过调用INT 15h
BIOS中断,递增di的值(20的倍数),让BIOS帮我们查找出一个一个的内存布局entry,并放入到一个保存地址范围描述符结构的缓冲区中,供后续的ucore进一步进行物理内存管理。这个缓冲区结构定义在memlayout.h中:

\begin{lstlisting}
struct e820map {
    int nr_map;
    struct {
        long long addr;
        long long size;
        long type;
    } map[E820MAX];
};
\end{lstlisting}

