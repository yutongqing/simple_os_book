\subsection{操作系统接口}

首先,读者可站在使用操作系统的角度来看操作系统。操作系统内核是一个需要提供各种服务的软件,其服务对象是应用程序,而用户(这里可以理解为一般使用计算机的人)是通过应用程序的服务间接获得操作系统的服务的),所以操作系统内核藏在一般用户看不到的地方。但应用程序需要访问操作系统,获得操作系统的服务,这就需要通过操作系统的接口才能完成。如果把操作系统看成是一个函数库,那么其接口就是函数名称和它的参数。但操作系统不是简单的一个函数库,它的接口需要考虑安全因素,使得应用软件不能直接读写操作系统内部函数的地址地址空间,为此,操作系统设计了一个安全可靠的接口,我们称为系统调用接口(System Call Interface),应用程序可以通过系统调用接口请求获得操作系统的服务,但不能直接调用操作系统的函数和全局变量;操作系统提供完服务后,返回应用程序继续执行。

对于实际操作系统而言,具有大量的服务接口,比如Linux有上百个系统调用接口。为了简单起见,以ucore OS为例,可以看到它为应用程序提供了如下一些接口:

\begin{enumerate}
	\item 
	\item  进程管理:复制创建--fork、退出--exit、执行--exec、...
	\item  同步互斥的并发控制:信号量--semaphore、管程--monitor、条件变量--condition variable 、...
	\item  进程间通信:管道--pipe、信号--signal、事件--event、邮箱--mailbox、共享内存--shared mem、...
	\item  文件I/O操作:读--read、写--write、打开--open、关闭--close、...
	\item  外设I/O操作:外设包括键盘、显示器、串口、磁盘、时钟、...,但接口是直接采用了文件I/O操作的系统调用接口
\end{enumerate}


这在某种程度上说明了文件是外设的一种抽象。在UNIX中(ucore是模仿UNIX),大部分外设都可以以文件的形式来访问。
有了这些接口,简单的应用程序就不用考虑底层硬件细节,可以在操作系统的服务支持和管理下简洁地完成其应用功能了。






